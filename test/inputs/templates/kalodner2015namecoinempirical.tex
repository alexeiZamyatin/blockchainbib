\section{An empirical study of Namecoin and lessons for decentralized namespace design}
\bibentry{kalodner2015namecoinempirical}

\textbf{Abstract:} 
Secure decentralized namespaces have recently become possible due to cryptocurrency technology. They enable a censorship-resistant domainname system outside the control of any single entity, among other applications. Namecoin, a fork of Bitcoin, is the most prominent example. We initiate the study of decentralized namespaces and the market for names in such systems. Our extensive empirical analysis of Namecoin reveals a system in disrepair. Indeed, our methodology for detecting ''squatted'' and otherwise inactive domains reveals that among Namecoin’s roughly 120,000 registered domain names, a mere 28 are not squatted and have nontrivial content. Further, we develop techniques for detecting transfers of domains in the Namecoin block chain and provide evidence that the market for domains is thin-tononexistent. We argue that the state of the art in mechanism design for decentralized namespace markets is lacking. We propose a model of utility of different names to different participants, and articulate desiderata of a decentralized namespace in terms of this utility function. We use this model to explore the design
