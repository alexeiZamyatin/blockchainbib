\section{Bringing Deployable Key Transparency to End Users }
\bibentry{melara2015keytransparency}

\textbf{Abstract:} 
We present CONIKS, an end-user key verification service capable of integration in end-to-end encrypted communication systems. CONIKS builds on related designs for transparency of web server certificates but solves several new challenges specific to key verification for end users. In comparison to prior designs, CONIKS enables more efficient monitoring and auditing of keys, allowing small organizations to effectively audit even very large key servers. CONIKS users can efficiently monitor their own key bindings for consistency, downloading less than 20 kB per day to do so even for a provider with billions of users. CONIKS users and providers can collectively audit providers for non-equivocation, and this requires downloading a constant 2.5 kB per day regardless of server size. Unlike any previous proposal, CONIKS also preserves the level of privacy offered by today’s major communication services, hiding the list of usernames present and even allowing providers to conceal the total number of users in the system.
