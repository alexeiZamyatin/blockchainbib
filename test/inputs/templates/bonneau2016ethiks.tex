\section{EthIKS: Using Ethereum to audit a CONIKS key transparency log}
\bibentry{bonneau2016ethiks}

\textbf{Abstract:} 
CONIKS is a proposed key transparency system which enables a centralized service provider to maintain an auditable yet privacypreserving directory of users' public keys. In  the original CONIKS design, users must monitor that their data is correctly included in every published snapshot of the directory, necessitating either slow updates or trust in an unspecied third-party to audit that the data structure has stayed consistent.  We  demonstrate  that  the  data  structures  for CONIKS  are very similar to those used in Ethereum, a consensus computation platform  with  a  Turing-complete  programming  environment.  We  can  take advantage  of  this  to  embed  the  core  CONIKS  data  structures  into  an Ethereum contract with only minor modications. Users may then trust the Ethereum network to audit the data structure for consistency and non-equivocation. Users who do not trust (or are unaware of) Ethereum can  self-audit  the  CONIKS  data  structure  as  before. We  have  implemented a prototype contract for our hybrid EthIKS scheme, demonstrating that it adds only modest bandwidth overhead to CONIKS proofs and costs hundredths of pennies per key update in fees at today's rates.
