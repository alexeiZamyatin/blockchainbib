\section{On Bitcoin as a public randomness source}
\bibentry{bonneau2015random}

\textbf{Abstract:} 
We formalize the use of Bitcoin as a source of publiclyverifiable randomness. As a side-effect of Bitcoin’s proof-of-work-based consensus system random values are broadcast every time new blocks are mined. We can derive strong lower bounds on the computational min-entropy in each block: currently at least 68 bits of min-entropy are produced every 10 minutes from which one can derive over 32 nearuniform bits using standard extractor techniques. We show that any attack on this beacon would form an attack on Bitcoin itself and hence have a monetary cost that we can bound unlike any other construction for a public randomness beacon in the literature. In our simplest construction we show that a lottery producing a single unbiased bit is manipulation-resistant against an attacker with a stake of less than 50 bitcoins in the output or about US\$12 000 today. Finally we propose making the beacon output available to smart contracts and demonstrate that this simple tool enables a number of interesting applications.
