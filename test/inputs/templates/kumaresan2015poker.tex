\section{How to Use Bitcoin to Play Decentralized Poker }
\bibentry{kumaresan2015poker}

\textbf{Abstract:} 
Back and Bentov (arXiv 2014) and Andrychowicz et al. (Security and Privacy 2014) introduced techniques to perform secure multiparty computations on Bitcoin. Among other things, these works constructed lottery protocols that ensure that any party that aborts after learning the outcome pays a monetary penalty to all other parties. Following this, Andrychowicz et al. (Bitcoin Workshop 2014) and concurrently Bentov and Kumaresan (Crypto 2014) extended the solution to arbitrary secure function
evaluation while guaranteeing fairness in the following sense: any party that aborts after learning the output pays a monetary penalty to all parties that did not learn the output. Andrychowicz et al. (Bitcoin Workshop 2014) also suggested extending to scenarios where parties receive a payoff according to the output of a secure function evaluation, and outlined a 2-party protocol for the same that in addition satisfies the notion of fairness described above. In this work, we formalize,
generalize, and construct multiparty protocols for the primitive suggested by Andrychowicz et al. We call this primitive secure cash distribution with penalties. Our formulation of secure cash distribution with penalties poses it as a multistage reactive functionality (i.e., more general than secure function evaluation) that provides a way to securely implement smart contracts in a decentralized setting, and consequently suffices to capture a wide variety of stateful computations involving data
and/or money, such as decentralized auctions, markets, and games such as poker, etc. Our protocol realizing secure cash distribution with penalties works in a hybrid model where parties have access to a claim-or-refund transaction functionality $F^{*}_{CR}$ which can be efficiently realized in (a variant of) Bitcoin, and is otherwise independent of the Bitcoin ecosystem. We emphasize that our protocol is dropout-tolerant in the sense that any party that drops out during the protocol is forced to pay a monetary penalty to all other parties. Our formalization and construction generalize both secure computation with penalties of Bentov and Kumaresan (Crypto 2014), and secure lottery with penalties of Andrychowicz et al. (Security and Privacy 2014).

