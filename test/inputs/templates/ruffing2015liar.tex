\section{Liar, Liar, Coins on Fire!: Penalizing Equivocation By Loss of Bitcoins }
\bibentry{ruffing2015liar}

\textbf{Abstract:} 
We show that equivocation, i.e., making conflicting statements to others in a distributed protocol, can be monetarily disincentivized by the use of crypto-currencies such as Bitcoin. To this end, we design completely decentralized non-equivocation contracts, which make it possible to penalize an equivocating party by the loss of its money. At the core of these contracts, there is a novel cryptographic primitive called accountable assertions, which reveals the party’s Bitcoin credentials if it equivocates. Non-equivocation contracts are particularly useful for distributed systems that employ public append-only logs to protect data integrity, e.g., in cloud storage and social networks. Moreover, as double-spending in Bitcoin is a special case of equivocation, the contracts enable us to design a payment protocol that allows a payee to receive funds at several unsynchronized points of sale, while being able to penalize a double-spending payer after the fact.
