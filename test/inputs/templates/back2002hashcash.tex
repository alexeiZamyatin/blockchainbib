\section{Hashcash-a denial of service counter-measure}
\bibentry{back2002hashcash}

\textbf{Abstract:} Hashcash was originally proposed as a mechanism to throttle systematic abuse of un-metered internet resources such as email, and anonymous remailers in May 1997. Five years on, this paper captures in one place the various applications, improvements suggested and related subsequent publications, and describes initial experience from experiments using hashcash. The hashcash CPU cost-function computes a token which can be used as a proof-of-work. Interactive and noninteractive variants of cost-functions can be constructed which can be used in situations where the server can issue a challenge (connection oriented interactive protocol), and where it can not (where the communication is store–and–forward, or packet oriented) respectively.
